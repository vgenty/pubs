
Now that we spent many pages to discuss about the {\pubs} model, let's talk
about something real and practical. This section presents a list of {\pubs} 
functions implemented on the {\psql} server. About a half of them are for
experts' use (in fact mostly for daemon and automated scripts since human
hands are one of last things to be trusted), and the other half is for
project scripts to use. 

If you are a project code developper and do not find a function of your
need, please contact the author and he will be more than happy to assist
how the existing function may solve the problem or implement a brand
new function to make your life easier.

\subsection{Project Information/Status Query}
These are functions that can be used by projects upon execution. That being
said, however, it is {\bf \color{blue} strongly recommended to use {\python} 
API within {\pubs} to execute these functions}. They should not be executed
from {\psql} interpreter or directly executing from an SQL script. If the 
list lacks any function needed for a project execution, please contact the 
author with a request. Functions and corresponding {\python} API will be 
provided.
\begin{itemize}
  \item {\bf DoesTableExist( name TEXT )}
    \begin{itemize}
      \item Checks if a table of the {\it name} exists or not in the database
        by checking the administrative master table. The table name is required
        to be in lowercase (there is no uppercase vs. lowercase in distinction
        among {\psql} server objects).
    \end{itemize}
  \item {\bf DoesProjectExist( name TEXT )}
    \begin{itemize}
      \item Checks if a project with the {\it name} exists or not in the 
        database. In addition to DoesTableExist(), this function checks
        if a specified project exists or not.
    \end{itemize}
  \item {\bf GetRunTimeStamp( Run INT, SubRun INT )}
    \begin{itemize}
      \item A function to retrieve the run start and end time stamp.
    \end{itemize}
  \item {\bf ProjectResource( name TEXT )}
    \begin{itemize}
      \item Returns a project resource (information needed for an execution)
        for a specified project name.
    \end{itemize}
  \item {\bf IncreaseProjSequence( name TEXT, run INT, subrun INT, nseq
    SMALLINT, status SMALLINT) }
    \begin{itemize}
      \item Increase number of sequence count in the specified project table
        for the specified run/sub-run number combination. Input status code 
        is used for all newly created TaskIDs.
    \end{itemize}
  \item {\bf UpdateProjStatus( name TEXT, run INT, subrun INT, seq SMALLINT,
    status SMALLINT, data TEXT)}
    \begin{itemize}
      \item Update the specified project's status for the specified TaskID. 
        At the same time, a TaskID specific data can be also stored although
        that is not necessary (by default the last argument is set to NULL).
    \end{itemize}
  \item {\bf GetProjectData( name TEXT, run INT, subrun INT, seq SMALLINT )}
    \begin{itemize}
      \item Retrieve project data for a specified TaskID. Only accessible to
        The data from the latest version number to avoid a confusion (and
        hence version number cannot be specified).
    \end{itemize}
  \item {\bf GetRuns( name TEXT, status SMALLINT) }
    \begin{itemize}
      \item Returns a table of TaskID (run, sub-run, seq., project-version) 
        for which the specified project carries the specified status code.
    \end{itemize}
  \item {\bf GetRuns( TEXT[]::ARRAY, SMALLINT[]::ARRAY )}
    \begin{itemize}
      \item Similar to GetRuns and it returns a table of run/sub-run number 
      combinations for which all specified projects in the first argument
      carry specified status code in the second argument. This function
      is useful to obtain a list of run/sub-run numbers across multiple
      project tables for specific combination of status code. Because 
      a sequence number is project dependent, it returns run/sub-run for
      which all belonging sequence status uniquely matches with the specified
      status code.
    \end{itemize}

\end{itemize}


\subsection{Functions For Project Management}
These are functions to be used by daemon process to maintaine/running the
projects. In principle these should not be used by a project execution. 
\begin{itemize}
  \item {\bf RemoveProject( name TEXT )} 
    \begin{itemize}
      \item Properly remove a project: drop a project table and remove the
        project information entry from the ProcessTable.
    \end{itemize}
  \item {\bf ListProject()}
    \begin{itemize}
      \item List all projects with the latest version number from ProcessTable.
    \end{itemize}
  \item {\bf ListEnabledProject()}
    \begin{itemize}
      \item List currently enabled project information with the latest version
        number from the ProcessTable.
    \end{itemize}
  \item {\bf DefineProject( name TEXT, command TEXT, frequency INT, email TEXT,
    start\_run INT, start\_subrun INT, resource HSTORE, enabled BOOLEAN )}
    \begin{itemize}
      \item A function to define a new project. It takes in project information
        and registers into the ProcessTable. It also calls {\bf MakeProjTable}
        function to create a project table.
    \end{itemize}
  \item {\bf MakeProjTable( name TEXT )}
    \begin{itemize}
      \item Function dedicated to create a project table. This function is
        to be called by {\bf DefineProject} and not to be called by hand!
    \end{itemize}
  \item {\bf UpdateProjectConfig( name TEXT, command TEXT, frequency INT, email
    TEXT, resource HSTORE, enabled BOOLEAN, version INT)}
    \begin{itemize}
      \item A function to alter and update project configuration. As seen in
        the function arguments, start run/sub-run number cannot be altered by
        design.
    \end{itemize}
  \item {\bf ProjectVersionUpdate( name TEXT, command TEXT, frequency INT,
    email TEXT, run INT, subrun INT, resource HSTORE, enable BOOLEAN)}
    \begin{itemize}
      \item Increment the project version number and store new project
        information. Unlike {\bf UpdateProjectConfig}, this function can
        register any project information as there will be a distinct row
        to be inserted in the ProcessTable.
    \end{itemize}
  \item {\bf GetVersionRunRange( name TEXT )}
    \begin{itemize}
      \item For a specified project name, returns multiple result sets each
        representing a specific run number range with the corresponding
        project version number.
    \end{itemize}
  \item {\bf InsertIntoProjTable( name TEXT, run INT, subrun INT )}
    \begin{itemize}
      \item Insert a new run/sub-run number entry into a project table with
        the default status code of 1. The latest version number for the
        subject run/sub-run is also taken from the ProcessTable.
    \end{itemize}
  \item {\bf OneProjectRunSynch()}
    \begin{itemize}
      \item Make sure one particular project table has run/sub-run 
        numbers that currently appears in the MainRun table and above
        the specified run/subrun numbers in the argument.
    \end{itemize}
  \item {\bf AllProjectRunSynch()}
    \begin{itemize}
      \item Make sure all project table has run/sub-run numbers that 
        currently appears in the MainRun and above the specified start
        run/sub-run numbers in the project information.
    \end{itemize}
  \item {\bf ProjectInfo(name TEXT, ver INT)}
    \begin{itemize}
      \item Returns project information for a specified version number.
        By default the version number does not need to be specified.
        If not given, it is set to the latest version number. This function
        is used to run a project via daemon.
    \end{itemize}
\end{itemize}

\subsection{Admin Functions}
Functions prepared for the top-level administrative purposes. These functions
should be executed by database admins only.
\begin{itemize}
  \item {\bf RemoveProcessDB()}
    \begin{itemize}
      \item ``Properly'' remove {\it everything}. This function drops all
        projects registered in ProcessTable using {\bf RemoveProject} 
        function. Then it drops an empty ProcessTable.
    \end{itemize}
  \item {\bf CreateProcessTable()}
    \begin{itemize}
      \item A simple function to create the ProcessTable.
    \end{itemize}
  \item {\bf CreateTestRunTable()}
    \begin{itemize}
      \item A function to create ``fake'' MainRun table. This is for 
        development work, and not for an official operation. In the official
        production, MainRun table is slave-copied from the configuration
        database automatically.
    \end{itemize}
  \item {\bf InsertIntoTestRunTable( Run INT, SubRun INT, TimeStart TIMESTAMP, TimeEnd TIMESTAMP )}
    \begin{itemize}
      \item A function to insert a new entry into the ``fake'' MainRun table.
        This is not meant to be used for the offial production.
    \end{itemize}
  \item {\bf FillTestRunTable( NRuns INT, NSubRuns INT)}
    \begin{itemize}
      \item A function to fill the ``fake'' MainRun table with multiple entries
        at once. It fills the table with NRuns, each with NSubRuns.
    \end{itemize}
  \item {\bf CheckDBIntegrity()}
    \begin{itemize}
      \item Returns a boolean after checking the process DB integrity. 
        In particular it checks if ProcessTable exists or not, and then
        checks if all projects registered in ProcessTable have own project
        tables.
    \end{itemize}
\end{itemize}



