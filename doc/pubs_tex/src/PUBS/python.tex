The ``software framework'' part of {\pubs}, which provides the code base 
for application development, is really in {\python}. This section describes
 {\python} tools in {\pubs} that can be used to develop a project execution 
code and set up the data processing chain of multiple projects.

There are three (somewhat) big {\python} modules in {\pubs}:
\begin{itemize}
  \item {\pubutil} $\ldots$ basic framework tools
  \item {\pubdbi} $\ldots$ generic database interface based using {\psycopg}
  \item {\dstream} $\ldots$ data processing framework toolkit
\end{itemize}
A project code developer interfaces with {\dstream} directly while that itself 
depends on basic tools defined in {\pubutil} and {\pubdbi}. We go over each 
of these in the following sections. 

\subsection{{\pubutil}}
This module introduce 3 objects: {\publogger}, {\pubsmtp}, and {\pubexception}.
They are framework logging tool, email sender function via SMTP, and a base exception
class definition. 

\subsubsection{{\publogger} $\ldots$ Logging Module}
This is the framework logger tool, and uses a popular {\python}'s logging module.
{\publogger} is a factory class that can instansiate an individual logger instance
with a specific message format and a choice of stream: either {\stdout}/{\stderr} or
output file stream. 

Each logger instance created by {\publogger} factory has a unique name, and 
can be instantiated by a factory function call:
\begin{lstlisting}
  >>> pubs_logger.get_logger('my_logger')
\end{lstlisting}
for a logger named ``my\_logger''. {\publogger} keeps track of all created loggers 
in its class variable {\ttfamily \_loggers}. When there is a request for a logger
with the same name created in the past, it returns the same instance.
\begin{lstlisting}
  >>> from pub_util import pub_logger
  >>> pub_logger.get_logger('a')
  [ INFO    ] pub_logger (L: 81 ) >> {_add_logger} OPENED LOGGER a
  <logging.Logger object at 0x10b903f90>
  >>> pub_logger.get_logger('a')
  <logging.Logger object at 0x10b903f90>
  >>>
\end{lstlisting}

As you might expect in any similar tool, {\publogger} has several message levels:
debug, info, warning, error, and critical. The default message level is set via
shell environment variable {\ttfamily \$PUB\_LOGGER\_LEVEL}, which is automatically
set in {\ttfamily setup.sh} configuration script. You may change the level if you
wish. You do not have to change the configuration script, but instead just change
the shell environment variable in any way you want (for instance by hand on your
terminal instead of sourcing a script). The set shell environment value is parsed 
in {\ttfamily pub\_util/pub\_env.py} script to an appropriate value. 
Similarly, the stream destination (either {\stdout} or file stream) is
set via shell environment variable {\ttfamily \$PUB\_LOGGER\_DRAIN}, again set
automatically in {\ttfamily setup.sh}. In case the drain is chosen to be a text
file stream, {\ttfamily \$PUB\_LOGGER\_FILE\_LOCATION} environment variable's value
is used as the log file location.

Each message level has a dedicated logger function call to parse an output message
through your logger. Here is an example of formatted output :
\begin{lstlisting}
  >>> a.debug('This is debug')
  [ DEBUG   ] <stdin> (L: 1  ) >> {<module>} This is debug
  >>> a.info('This is info')
  [ INFO    ] <stdin> (L: 1  ) >> {<module>} This is info
  >>> a.warning('This is warning')
  [ WARNING ] <stdin> (L: 1  ) >> {<module>} This is warning
  >>> a.error('This is error')
  [ ERROR   ] <stdin> (L: 1  ) >> {<module>} This is error
  >>> a.critical('This is critical')
  [ CRITICAL] <stdin> (L: 1  ) >> {<module>} This is critical
\end{lstlisting}
The logger specifies the message level, and prints out three more information in
addition to the sent message by the caller. The first `$<$stdin$>$' tells where the
message is sent from. `(L: 1  )' tells us which line in the caller's module code
this function is called from. Then `\{$<$module$>$\}' tells us the name of the
caller's module. In the above example, this is called from the main, and hence
it is not really useful. However, these information help us to track down problems
easily as you can identify where each function call is made. For instance, running
{\ttfamily ds\_daemon.py} to test the installation (see Sec.\ref{prep:pubs:daemon}),
you have probably seen this message:
\begin{lstlisting}
  [ DEBUG   ] ds_daemon (L: 128) >> {load_projects} Updating project dummy_daq ...
\end{lstlisting}
 This menas that a logger function ``debug'' was called by a function 
{\ttfamily load\_projects} and the exact location is in line number 128 of the 
module code {\ttfamily ds\_daemon.py}. 

\subsubsection{{\pubsmtp} $\ldots$ Simple SMTP Protocole}


\subsection{{\pubdbi}}

\subsection{{\dstream}}
