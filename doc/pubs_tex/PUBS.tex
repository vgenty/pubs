\documentclass[a4paper,12pt]{report}
\usepackage{lgrind}
\pagestyle{plain}
\usepackage{amsmath,amsfonts,amssymb,amsthm}
\usepackage{listings}
\usepackage{cancel}
\usepackage{appendix}
\usepackage{fancyhdr}   
\usepackage{url}
\usepackage{path}
\usepackage[dvipdfm]{graphicx}
\usepackage{color}
\usepackage{array}
\usepackage{authblk}
\usepackage[margin=2cm, head=2cm, foot=1cm]{geometry}
\usepackage{verbatim}
\setlength{\topmargin}{-2.5cm}
\setlength{\textwidth}{17cm}
\setlength{\oddsidemargin}{-0.5cm}
\setlength{\evensidemargin}{-0.5cm}
\setlength{\textheight}{22.5cm}
\setlength{\footskip}{1cm}
\renewcommand{\topfraction}{0.9}
\renewcommand{\bottomfraction}{0.8}
\setcounter{topnumber}{2}
\setcounter{bottomnumber}{2}
\setcounter{totalnumber}{4}
\usepackage{hyperref}
\usepackage{pdfpages}
\lstset{columns=fullflexible,basicstyle=\ttfamily}


\renewcommand{\floatpagefraction}{0.9}%

\newcommand{\ROOT}{{\ttfamily ROOT}}
\newcommand{\CINT}{{\ttfamily CINT}}
\newcommand{\CPP}{{\ttfamily C++}}
\newcommand{\gpp}{{\ttfamily g++}}
\newcommand{\clang}{{\ttfamily clang}}
\newcommand{\php}{{\ttfamily php}}
\newcommand{\sql}{{\ttfamily SQL}}
\newcommand{\psql}{{\ttfamily PostgreSQL}}
\newcommand{\python}{{\ttfamily python}}
\newcommand{\PyROOT}{{\ttfamily PyROOT}}
\newcommand{\larlight}{{\ttfamily LArLight}}
\newcommand{\larsoft}{{\ttfamily LArSoft}}
\newcommand{\git}{{\ttfamily git}}
\newcommand{\doxygen}{{\ttfamily doxygen}}
\newcommand{\HTML}{{\ttfamily HTML}}
\newcommand{\ART}{{\ttfamily ART}}
\newcommand{\enum}{{\ttfamily enum}}
\newcommand{\psycopg}{{\ttfamily psycopg2}}
\newcommand{\aptget}{{\ttfamily apt-get}}
\newcommand{\yum}{{\ttfamily yum}}
\newcommand{\fink}{{\ttfamily fink}}
\newcommand{\brew}{{\ttfamily home brew}}
\newcommand{\macport}{{\ttfamily macport}}
\newcommand{\xcode}{{\ttfamily Xcode}}
\newcommand{\stdout}{{\ttfamily stdout}}
\newcommand{\stderr}{{\ttfamily stderr}}
\newcommand{\hstore}{{\ttfamily HSTORE}}

\newcommand{\pubs}{{\ttfamily pubs}}
\newcommand{\procdb}{{\ttfamily procdb}}
\newcommand{\dstream}{{\ttfamily dstream}}

\newcommand{\pubutil}{{\ttfamily pub\_util}}
\newcommand{\pubsmtp}{{\ttfamily pub\_smtp}}
\newcommand{\publogger}{{\ttfamily pub\_logger}}
\newcommand{\pubexception}{{\ttfamily pub\_exception}}

\newcommand{\pubdbi}{{\ttfamily pub\_dbi}}
\newcommand{\pubdbconn}{{\ttfamily pubdb\_conn}}
\newcommand{\pubdbdata}{{\ttfamily pubdb\_data}}
\newcommand{\pubdbexception}{{\ttfamily pubdb\_exception}}
\newcommand{\pubdbapi}{{\ttfamily pubdb\_api}}
\newcommand{\pubdbreader}{{\ttfamily pubdb\_reader}}
\newcommand{\pubdbwriter}{{\ttfamily pubdb\_writer}}





\begin{document}

\title{PUBS: Data Processing Software Framework}
%\date{March 12th, 2014}
\author{
  Kazuhiro Terao, kazuhiro@nevis.columbia.edu
}
\maketitle

\begin{abstract}
This document describes about {\python/\php/\psql} UBooNE Software (a.k.a. \pubs) framework
written for online data processing of the experiment. Each step of data processing, defined
as a ``project'', carries a project-specific category of {\it state} and corresponding 
actions to deliver data to a next step. Projects' state are stored in {\psql} database server
with a dedicated procedures for data queries. {\pubs} provides a generic, {\python} based software 
framework for development of ``project'' code. In addition, it provides necessary tools for
project execution including a daemon program for parallel project execution and framework logger
system. Monitoring of project execution is realized through a web-browser using a {\php} based 
toolkit, which can be also used for experts to actively interface with currently running projects.
Finally this document assumes you have a modest knowledge about {\python} and common programming 
terminologies. 
\end{abstract}

% Table of Contents
\tableofcontents

\newpage

% Prep
\chapter{Preparation}
\label{prep}

This chapter describes about building blocks of {\pubs} framework.
Three sections cover {\psql}, {\python}, and {\php} component respectively.
We do not cover in details about {\dstream} which is a particular application
of {\pubs}. Instead {\dstream} is covered in the next chapter.

\section{{\python} Software Framework}
The ``software framework'' part of {\pubs}, which provides the code base 
for application development, is really in {\python}. This section describes
 {\python} tools in {\pubs} that can be used to develop a project execution 
code and set up the data processing chain of multiple projects.

There are three (somewhat) big {\python} modules in {\pubs}:
\begin{itemize}
  \item {\pubutil} $\ldots$ basic framework tools
  \item {\pubdbi} $\ldots$ generic database interface based using {\psycopg}
  \item {\dstream} $\ldots$ data processing framework toolkit
\end{itemize}
A project code developer interfaces with {\dstream} directly while that itself 
depends on basic tools defined in {\pubutil} and {\pubdbi}. We go over each 
of these in the following sections. 

\subsection{{\pubutil}}
This module introduce 3 objects: {\publogger}, {\pubsmtp}, and {\pubexception}.
They are framework logging tool, email sender function via SMTP, and a base exception
class definition. 

\subsubsection{{\publogger} $\ldots$ Logging Module}
This is the framework logger tool, and uses a popular {\python}'s logging module.
{\publogger} is a factory class that can instansiate an individual logger instance
with a specific message format and a choice of stream: either {\stdout}/{\stderr} or
output file stream. 

Each logger instance created by {\publogger} factory has a unique name, and 
can be instantiated by a factory function call:
\begin{lstlisting}
  >>> pubs_logger.get_logger('my_logger')
\end{lstlisting}
for a logger named ``my\_logger''. {\publogger} keeps track of all created loggers 
in its class variable {\ttfamily \_loggers}. When there is a request for a logger
with the same name created in the past, it returns the same instance.
\begin{lstlisting}
  >>> from pub_util import pub_logger
  >>> pub_logger.get_logger('a')
  [ INFO    ] pub_logger (L: 81 ) >> {_add_logger} OPENED LOGGER a
  <logging.Logger object at 0x10b903f90>
  >>> pub_logger.get_logger('a')
  <logging.Logger object at 0x10b903f90>
  >>>
\end{lstlisting}

As you might expect in any similar tool, {\publogger} has several message levels:
debug, info, warning, error, and critical. The default message level is set via
shell environment variable {\ttfamily \$PUB\_LOGGER\_LEVEL}, which is automatically
set in {\ttfamily setup.sh} configuration script. You may change the level if you
wish. You do not have to change the configuration script, but instead just change
the shell environment variable in any way you want (for instance by hand on your
terminal instead of sourcing a script). The set shell environment value is parsed 
in {\ttfamily pub\_util/pub\_env.py} script to an appropriate value. 
Similarly, the stream destination (either {\stdout} or file stream) is
set via shell environment variable {\ttfamily \$PUB\_LOGGER\_DRAIN}, again set
automatically in {\ttfamily setup.sh}. In case the drain is chosen to be a text
file stream, {\ttfamily \$PUB\_LOGGER\_FILE\_LOCATION} environment variable's value
is used as the log file location.

Each message level has a dedicated logger function call to parse an output message
through your logger. Here is an example of formatted output :
\begin{lstlisting}
  >>> a.debug('This is debug')
  [ DEBUG   ] <stdin> (L: 1  ) >> {<module>} This is debug
  >>> a.info('This is info')
  [ INFO    ] <stdin> (L: 1  ) >> {<module>} This is info
  >>> a.warning('This is warning')
  [ WARNING ] <stdin> (L: 1  ) >> {<module>} This is warning
  >>> a.error('This is error')
  [ ERROR   ] <stdin> (L: 1  ) >> {<module>} This is error
  >>> a.critical('This is critical')
  [ CRITICAL] <stdin> (L: 1  ) >> {<module>} This is critical
\end{lstlisting}
The logger specifies the message level, and prints out three more information in
addition to the sent message by the caller. The first `$<$stdin$>$' tells where the
message is sent from. `(L: 1  )' tells us which line in the caller's module code
this function is called from. Then `\{$<$module$>$\}' tells us the name of the
caller's module. In the above example, this is called from the main, and hence
it is not really useful. However, these information help us to track down problems
easily as you can identify where each function call is made. For instance, running
{\ttfamily ds\_daemon.py} to test the installation (see Sec.\ref{prep:pubs:daemon}),
you have probably seen this message:
\begin{lstlisting}
  [ DEBUG   ] ds_daemon (L: 128) >> {load_projects} Updating project dummy_daq ...
\end{lstlisting}
 This menas that a logger function ``debug'' was called by a function 
{\ttfamily load\_projects} and the exact location is in line number 128 of the 
module code {\ttfamily ds\_daemon.py}. 

\subsubsection{{\pubsmtp} $\ldots$ Simple SMTP Protocole}


\subsection{{\pubdbi}}

\subsection{{\dstream}}


\section{{\psql} Database Schema And Functions}

Now that we spent many pages to discuss about the {\pubs} model, let's talk
about something real and practical. This section presents a list of {\pubs} 
functions implemented on the {\psql} server. About a half of them are for
experts' use (in fact mostly for daemon and automated scripts since human
hands are one of last things to be trusted), and the other half is for
project scripts to use. 

If you are a project code developper and do not find a function of your
need, please contact the author and he will be more than happy to assist
how the existing function may solve the problem or implement a brand
new function to make your life easier.

\subsection{Project Information/Status Query}
These are functions that can be used by projects upon execution. That being
said, however, it is {\bf \color{blue} strongly recommended to use {\python} 
API within {\pubs} to execute these functions}. They should not be executed
from {\psql} interpreter or directly executing from an SQL script. If the 
list lacks any function needed for a project execution, please contact the 
author with a request. Functions and corresponding {\python} API will be 
provided.
\begin{itemize}
  \item {\bf DoesTableExist( name TEXT )}
    \begin{itemize}
      \item Checks if a table of the {\it name} exists or not in the database
        by checking the administrative master table. The table name is required
        to be in lowercase (there is no uppercase vs. lowercase in distinction
        among {\psql} server objects).
    \end{itemize}
  \item {\bf DoesProjectExist( name TEXT )}
    \begin{itemize}
      \item Checks if a project with the {\it name} exists or not in the 
        database. In addition to DoesTableExist(), this function checks
        if a specified project exists or not.
    \end{itemize}
  \item {\bf GetRunTimeStamp( Run INT, SubRun INT )}
    \begin{itemize}
      \item A function to retrieve the run start and end time stamp.
    \end{itemize}
  \item {\bf ProjectResource( name TEXT )}
    \begin{itemize}
      \item Returns a project resource (information needed for an execution)
        for a specified project name.
    \end{itemize}
  \item {\bf IncreaseProjSequence( name TEXT, run INT, subrun INT, nseq
    SMALLINT, status SMALLINT) }
    \begin{itemize}
      \item Increase number of sequence count in the specified project table
        for the specified run/sub-run number combination. Input status code 
        is used for all newly created TaskIDs.
    \end{itemize}
  \item {\bf UpdateProjStatus( name TEXT, run INT, subrun INT, seq SMALLINT,
    status SMALLINT, data TEXT)}
    \begin{itemize}
      \item Update the specified project's status for the specified TaskID. 
        At the same time, a TaskID specific data can be also stored although
        that is not necessary (by default the last argument is set to NULL).
    \end{itemize}
  \item {\bf GetProjectData( name TEXT, run INT, subrun INT, seq SMALLINT )}
    \begin{itemize}
      \item Retrieve project data for a specified TaskID. Only accessible to
        The data from the latest version number to avoid a confusion (and
        hence version number cannot be specified).
    \end{itemize}
  \item {\bf GetRuns( name TEXT, status SMALLINT) }
    \begin{itemize}
      \item Returns a table of TaskID (run, sub-run, seq., project-version) 
        for which the specified project carries the specified status code.
    \end{itemize}
  \item {\bf GetRuns( TEXT[]::ARRAY, SMALLINT[]::ARRAY )}
    \begin{itemize}
      \item Similar to GetRuns and it returns a table of run/sub-run number 
      combinations for which all specified projects in the first argument
      carry specified status code in the second argument. This function
      is useful to obtain a list of run/sub-run numbers across multiple
      project tables for specific combination of status code. Because 
      a sequence number is project dependent, it returns run/sub-run for
      which all belonging sequence status uniquely matches with the specified
      status code.
    \end{itemize}

\end{itemize}


\subsection{Functions For Project Management}
These are functions to be used by daemon process to maintaine/running the
projects. In principle these should not be used by a project execution. 
\begin{itemize}
  \item {\bf RemoveProject( name TEXT )} 
    \begin{itemize}
      \item Properly remove a project: drop a project table and remove the
        project information entry from the ProcessTable.
    \end{itemize}
  \item {\bf ListProject()}
    \begin{itemize}
      \item List all projects with the latest version number from ProcessTable.
    \end{itemize}
  \item {\bf ListEnabledProject()}
    \begin{itemize}
      \item List currently enabled project information with the latest version
        number from the ProcessTable.
    \end{itemize}
  \item {\bf DefineProject( name TEXT, command TEXT, frequency INT, email TEXT,
    start\_run INT, start\_subrun INT, resource HSTORE, enabled BOOLEAN )}
    \begin{itemize}
      \item A function to define a new project. It takes in project information
        and registers into the ProcessTable. It also calls {\bf MakeProjTable}
        function to create a project table.
    \end{itemize}
  \item {\bf MakeProjTable( name TEXT )}
    \begin{itemize}
      \item Function dedicated to create a project table. This function is
        to be called by {\bf DefineProject} and not to be called by hand!
    \end{itemize}
  \item {\bf UpdateProjectConfig( name TEXT, command TEXT, frequency INT, email
    TEXT, resource HSTORE, enabled BOOLEAN, version INT)}
    \begin{itemize}
      \item A function to alter and update project configuration. As seen in
        the function arguments, start run/sub-run number cannot be altered by
        design.
    \end{itemize}
  \item {\bf ProjectVersionUpdate( name TEXT, command TEXT, frequency INT,
    email TEXT, run INT, subrun INT, resource HSTORE, enable BOOLEAN)}
    \begin{itemize}
      \item Increment the project version number and store new project
        information. Unlike {\bf UpdateProjectConfig}, this function can
        register any project information as there will be a distinct row
        to be inserted in the ProcessTable.
    \end{itemize}
  \item {\bf GetVersionRunRange( name TEXT )}
    \begin{itemize}
      \item For a specified project name, returns multiple result sets each
        representing a specific run number range with the corresponding
        project version number.
    \end{itemize}
  \item {\bf InsertIntoProjTable( name TEXT, run INT, subrun INT )}
    \begin{itemize}
      \item Insert a new run/sub-run number entry into a project table with
        the default status code of 1. The latest version number for the
        subject run/sub-run is also taken from the ProcessTable.
    \end{itemize}
  \item {\bf OneProjectRunSynch()}
    \begin{itemize}
      \item Make sure one particular project table has run/sub-run 
        numbers that currently appears in the MainRun table and above
        the specified run/subrun numbers in the argument.
    \end{itemize}
  \item {\bf AllProjectRunSynch()}
    \begin{itemize}
      \item Make sure all project table has run/sub-run numbers that 
        currently appears in the MainRun and above the specified start
        run/sub-run numbers in the project information.
    \end{itemize}
  \item {\bf ProjectInfo(name TEXT, ver INT)}
    \begin{itemize}
      \item Returns project information for a specified version number.
        By default the version number does not need to be specified.
        If not given, it is set to the latest version number. This function
        is used to run a project via daemon.
    \end{itemize}
\end{itemize}

\subsection{Admin Functions}
Functions prepared for the top-level administrative purposes. These functions
should be executed by database admins only.
\begin{itemize}
  \item {\bf RemoveProcessDB()}
    \begin{itemize}
      \item ``Properly'' remove {\it everything}. This function drops all
        projects registered in ProcessTable using {\bf RemoveProject} 
        function. Then it drops an empty ProcessTable.
    \end{itemize}
  \item {\bf CreateProcessTable()}
    \begin{itemize}
      \item A simple function to create the ProcessTable.
    \end{itemize}
  \item {\bf CreateTestRunTable()}
    \begin{itemize}
      \item A function to create ``fake'' MainRun table. This is for 
        development work, and not for an official operation. In the official
        production, MainRun table is slave-copied from the configuration
        database automatically.
    \end{itemize}
  \item {\bf InsertIntoTestRunTable( Run INT, SubRun INT, TimeStart TIMESTAMP, TimeEnd TIMESTAMP )}
    \begin{itemize}
      \item A function to insert a new entry into the ``fake'' MainRun table.
        This is not meant to be used for the offial production.
    \end{itemize}
  \item {\bf FillTestRunTable( NRuns INT, NSubRuns INT)}
    \begin{itemize}
      \item A function to fill the ``fake'' MainRun table with multiple entries
        at once. It fills the table with NRuns, each with NSubRuns.
    \end{itemize}
  \item {\bf CheckDBIntegrity()}
    \begin{itemize}
      \item Returns a boolean after checking the process DB integrity. 
        In particular it checks if ProcessTable exists or not, and then
        checks if all projects registered in ProcessTable have own project
        tables.
    \end{itemize}
\end{itemize}





\section{{\php}-based Web Interface}
\input{./src/PUBS/php}




% Introduction
\chapter{Basic of {\pubs}}
\label{pubs}

This chapter describes about building blocks of {\pubs} framework.
Three sections cover {\psql}, {\python}, and {\php} component respectively.
We do not cover in details about {\dstream} which is a particular application
of {\pubs}. Instead {\dstream} is covered in the next chapter.

\section{{\python} Software Framework}
The ``software framework'' part of {\pubs}, which provides the code base 
for application development, is really in {\python}. This section describes
 {\python} tools in {\pubs} that can be used to develop a project execution 
code and set up the data processing chain of multiple projects.

There are three (somewhat) big {\python} modules in {\pubs}:
\begin{itemize}
  \item {\pubutil} $\ldots$ basic framework tools
  \item {\pubdbi} $\ldots$ generic database interface based using {\psycopg}
  \item {\dstream} $\ldots$ data processing framework toolkit
\end{itemize}
A project code developer interfaces with {\dstream} directly while that itself 
depends on basic tools defined in {\pubutil} and {\pubdbi}. We go over each 
of these in the following sections. 

\subsection{{\pubutil}}
This module introduce 3 objects: {\publogger}, {\pubsmtp}, and {\pubexception}.
They are framework logging tool, email sender function via SMTP, and a base exception
class definition. 

\subsubsection{{\publogger} $\ldots$ Logging Module}
This is the framework logger tool, and uses a popular {\python}'s logging module.
{\publogger} is a factory class that can instansiate an individual logger instance
with a specific message format and a choice of stream: either {\stdout}/{\stderr} or
output file stream. 

Each logger instance created by {\publogger} factory has a unique name, and 
can be instantiated by a factory function call:
\begin{lstlisting}
  >>> pubs_logger.get_logger('my_logger')
\end{lstlisting}
for a logger named ``my\_logger''. {\publogger} keeps track of all created loggers 
in its class variable {\ttfamily \_loggers}. When there is a request for a logger
with the same name created in the past, it returns the same instance.
\begin{lstlisting}
  >>> from pub_util import pub_logger
  >>> pub_logger.get_logger('a')
  [ INFO    ] pub_logger (L: 81 ) >> {_add_logger} OPENED LOGGER a
  <logging.Logger object at 0x10b903f90>
  >>> pub_logger.get_logger('a')
  <logging.Logger object at 0x10b903f90>
  >>>
\end{lstlisting}

As you might expect in any similar tool, {\publogger} has several message levels:
debug, info, warning, error, and critical. The default message level is set via
shell environment variable {\ttfamily \$PUB\_LOGGER\_LEVEL}, which is automatically
set in {\ttfamily setup.sh} configuration script. You may change the level if you
wish. You do not have to change the configuration script, but instead just change
the shell environment variable in any way you want (for instance by hand on your
terminal instead of sourcing a script). The set shell environment value is parsed 
in {\ttfamily pub\_util/pub\_env.py} script to an appropriate value. 
Similarly, the stream destination (either {\stdout} or file stream) is
set via shell environment variable {\ttfamily \$PUB\_LOGGER\_DRAIN}, again set
automatically in {\ttfamily setup.sh}. In case the drain is chosen to be a text
file stream, {\ttfamily \$PUB\_LOGGER\_FILE\_LOCATION} environment variable's value
is used as the log file location.

Each message level has a dedicated logger function call to parse an output message
through your logger. Here is an example of formatted output :
\begin{lstlisting}
  >>> a.debug('This is debug')
  [ DEBUG   ] <stdin> (L: 1  ) >> {<module>} This is debug
  >>> a.info('This is info')
  [ INFO    ] <stdin> (L: 1  ) >> {<module>} This is info
  >>> a.warning('This is warning')
  [ WARNING ] <stdin> (L: 1  ) >> {<module>} This is warning
  >>> a.error('This is error')
  [ ERROR   ] <stdin> (L: 1  ) >> {<module>} This is error
  >>> a.critical('This is critical')
  [ CRITICAL] <stdin> (L: 1  ) >> {<module>} This is critical
\end{lstlisting}
The logger specifies the message level, and prints out three more information in
addition to the sent message by the caller. The first `$<$stdin$>$' tells where the
message is sent from. `(L: 1  )' tells us which line in the caller's module code
this function is called from. Then `\{$<$module$>$\}' tells us the name of the
caller's module. In the above example, this is called from the main, and hence
it is not really useful. However, these information help us to track down problems
easily as you can identify where each function call is made. For instance, running
{\ttfamily ds\_daemon.py} to test the installation (see Sec.\ref{prep:pubs:daemon}),
you have probably seen this message:
\begin{lstlisting}
  [ DEBUG   ] ds_daemon (L: 128) >> {load_projects} Updating project dummy_daq ...
\end{lstlisting}
 This menas that a logger function ``debug'' was called by a function 
{\ttfamily load\_projects} and the exact location is in line number 128 of the 
module code {\ttfamily ds\_daemon.py}. 

\subsubsection{{\pubsmtp} $\ldots$ Simple SMTP Protocole}


\subsection{{\pubdbi}}

\subsection{{\dstream}}


\section{{\psql} Database Schema And Functions}

Now that we spent many pages to discuss about the {\pubs} model, let's talk
about something real and practical. This section presents a list of {\pubs} 
functions implemented on the {\psql} server. About a half of them are for
experts' use (in fact mostly for daemon and automated scripts since human
hands are one of last things to be trusted), and the other half is for
project scripts to use. 

If you are a project code developper and do not find a function of your
need, please contact the author and he will be more than happy to assist
how the existing function may solve the problem or implement a brand
new function to make your life easier.

\subsection{Project Information/Status Query}
These are functions that can be used by projects upon execution. That being
said, however, it is {\bf \color{blue} strongly recommended to use {\python} 
API within {\pubs} to execute these functions}. They should not be executed
from {\psql} interpreter or directly executing from an SQL script. If the 
list lacks any function needed for a project execution, please contact the 
author with a request. Functions and corresponding {\python} API will be 
provided.
\begin{itemize}
  \item {\bf DoesTableExist( name TEXT )}
    \begin{itemize}
      \item Checks if a table of the {\it name} exists or not in the database
        by checking the administrative master table. The table name is required
        to be in lowercase (there is no uppercase vs. lowercase in distinction
        among {\psql} server objects).
    \end{itemize}
  \item {\bf DoesProjectExist( name TEXT )}
    \begin{itemize}
      \item Checks if a project with the {\it name} exists or not in the 
        database. In addition to DoesTableExist(), this function checks
        if a specified project exists or not.
    \end{itemize}
  \item {\bf GetRunTimeStamp( Run INT, SubRun INT )}
    \begin{itemize}
      \item A function to retrieve the run start and end time stamp.
    \end{itemize}
  \item {\bf ProjectResource( name TEXT )}
    \begin{itemize}
      \item Returns a project resource (information needed for an execution)
        for a specified project name.
    \end{itemize}
  \item {\bf IncreaseProjSequence( name TEXT, run INT, subrun INT, nseq
    SMALLINT, status SMALLINT) }
    \begin{itemize}
      \item Increase number of sequence count in the specified project table
        for the specified run/sub-run number combination. Input status code 
        is used for all newly created TaskIDs.
    \end{itemize}
  \item {\bf UpdateProjStatus( name TEXT, run INT, subrun INT, seq SMALLINT,
    status SMALLINT, data TEXT)}
    \begin{itemize}
      \item Update the specified project's status for the specified TaskID. 
        At the same time, a TaskID specific data can be also stored although
        that is not necessary (by default the last argument is set to NULL).
    \end{itemize}
  \item {\bf GetProjectData( name TEXT, run INT, subrun INT, seq SMALLINT )}
    \begin{itemize}
      \item Retrieve project data for a specified TaskID. Only accessible to
        The data from the latest version number to avoid a confusion (and
        hence version number cannot be specified).
    \end{itemize}
  \item {\bf GetRuns( name TEXT, status SMALLINT) }
    \begin{itemize}
      \item Returns a table of TaskID (run, sub-run, seq., project-version) 
        for which the specified project carries the specified status code.
    \end{itemize}
  \item {\bf GetRuns( TEXT[]::ARRAY, SMALLINT[]::ARRAY )}
    \begin{itemize}
      \item Similar to GetRuns and it returns a table of run/sub-run number 
      combinations for which all specified projects in the first argument
      carry specified status code in the second argument. This function
      is useful to obtain a list of run/sub-run numbers across multiple
      project tables for specific combination of status code. Because 
      a sequence number is project dependent, it returns run/sub-run for
      which all belonging sequence status uniquely matches with the specified
      status code.
    \end{itemize}

\end{itemize}


\subsection{Functions For Project Management}
These are functions to be used by daemon process to maintaine/running the
projects. In principle these should not be used by a project execution. 
\begin{itemize}
  \item {\bf RemoveProject( name TEXT )} 
    \begin{itemize}
      \item Properly remove a project: drop a project table and remove the
        project information entry from the ProcessTable.
    \end{itemize}
  \item {\bf ListProject()}
    \begin{itemize}
      \item List all projects with the latest version number from ProcessTable.
    \end{itemize}
  \item {\bf ListEnabledProject()}
    \begin{itemize}
      \item List currently enabled project information with the latest version
        number from the ProcessTable.
    \end{itemize}
  \item {\bf DefineProject( name TEXT, command TEXT, frequency INT, email TEXT,
    start\_run INT, start\_subrun INT, resource HSTORE, enabled BOOLEAN )}
    \begin{itemize}
      \item A function to define a new project. It takes in project information
        and registers into the ProcessTable. It also calls {\bf MakeProjTable}
        function to create a project table.
    \end{itemize}
  \item {\bf MakeProjTable( name TEXT )}
    \begin{itemize}
      \item Function dedicated to create a project table. This function is
        to be called by {\bf DefineProject} and not to be called by hand!
    \end{itemize}
  \item {\bf UpdateProjectConfig( name TEXT, command TEXT, frequency INT, email
    TEXT, resource HSTORE, enabled BOOLEAN, version INT)}
    \begin{itemize}
      \item A function to alter and update project configuration. As seen in
        the function arguments, start run/sub-run number cannot be altered by
        design.
    \end{itemize}
  \item {\bf ProjectVersionUpdate( name TEXT, command TEXT, frequency INT,
    email TEXT, run INT, subrun INT, resource HSTORE, enable BOOLEAN)}
    \begin{itemize}
      \item Increment the project version number and store new project
        information. Unlike {\bf UpdateProjectConfig}, this function can
        register any project information as there will be a distinct row
        to be inserted in the ProcessTable.
    \end{itemize}
  \item {\bf GetVersionRunRange( name TEXT )}
    \begin{itemize}
      \item For a specified project name, returns multiple result sets each
        representing a specific run number range with the corresponding
        project version number.
    \end{itemize}
  \item {\bf InsertIntoProjTable( name TEXT, run INT, subrun INT )}
    \begin{itemize}
      \item Insert a new run/sub-run number entry into a project table with
        the default status code of 1. The latest version number for the
        subject run/sub-run is also taken from the ProcessTable.
    \end{itemize}
  \item {\bf OneProjectRunSynch()}
    \begin{itemize}
      \item Make sure one particular project table has run/sub-run 
        numbers that currently appears in the MainRun table and above
        the specified run/subrun numbers in the argument.
    \end{itemize}
  \item {\bf AllProjectRunSynch()}
    \begin{itemize}
      \item Make sure all project table has run/sub-run numbers that 
        currently appears in the MainRun and above the specified start
        run/sub-run numbers in the project information.
    \end{itemize}
  \item {\bf ProjectInfo(name TEXT, ver INT)}
    \begin{itemize}
      \item Returns project information for a specified version number.
        By default the version number does not need to be specified.
        If not given, it is set to the latest version number. This function
        is used to run a project via daemon.
    \end{itemize}
\end{itemize}

\subsection{Admin Functions}
Functions prepared for the top-level administrative purposes. These functions
should be executed by database admins only.
\begin{itemize}
  \item {\bf RemoveProcessDB()}
    \begin{itemize}
      \item ``Properly'' remove {\it everything}. This function drops all
        projects registered in ProcessTable using {\bf RemoveProject} 
        function. Then it drops an empty ProcessTable.
    \end{itemize}
  \item {\bf CreateProcessTable()}
    \begin{itemize}
      \item A simple function to create the ProcessTable.
    \end{itemize}
  \item {\bf CreateTestRunTable()}
    \begin{itemize}
      \item A function to create ``fake'' MainRun table. This is for 
        development work, and not for an official operation. In the official
        production, MainRun table is slave-copied from the configuration
        database automatically.
    \end{itemize}
  \item {\bf InsertIntoTestRunTable( Run INT, SubRun INT, TimeStart TIMESTAMP, TimeEnd TIMESTAMP )}
    \begin{itemize}
      \item A function to insert a new entry into the ``fake'' MainRun table.
        This is not meant to be used for the offial production.
    \end{itemize}
  \item {\bf FillTestRunTable( NRuns INT, NSubRuns INT)}
    \begin{itemize}
      \item A function to fill the ``fake'' MainRun table with multiple entries
        at once. It fills the table with NRuns, each with NSubRuns.
    \end{itemize}
  \item {\bf CheckDBIntegrity()}
    \begin{itemize}
      \item Returns a boolean after checking the process DB integrity. 
        In particular it checks if ProcessTable exists or not, and then
        checks if all projects registered in ProcessTable have own project
        tables.
    \end{itemize}
\end{itemize}





\section{{\php}-based Web Interface}
\input{./src/PUBS/php}




% PUBS
\chapter{Data Processing Framework}
\label{dstream}

This chapter describes about building blocks of {\pubs} framework.
Three sections cover {\psql}, {\python}, and {\php} component respectively.
We do not cover in details about {\dstream} which is a particular application
of {\pubs}. Instead {\dstream} is covered in the next chapter.

\section{{\python} Software Framework}
The ``software framework'' part of {\pubs}, which provides the code base 
for application development, is really in {\python}. This section describes
 {\python} tools in {\pubs} that can be used to develop a project execution 
code and set up the data processing chain of multiple projects.

There are three (somewhat) big {\python} modules in {\pubs}:
\begin{itemize}
  \item {\pubutil} $\ldots$ basic framework tools
  \item {\pubdbi} $\ldots$ generic database interface based using {\psycopg}
  \item {\dstream} $\ldots$ data processing framework toolkit
\end{itemize}
A project code developer interfaces with {\dstream} directly while that itself 
depends on basic tools defined in {\pubutil} and {\pubdbi}. We go over each 
of these in the following sections. 

\subsection{{\pubutil}}
This module introduce 3 objects: {\publogger}, {\pubsmtp}, and {\pubexception}.
They are framework logging tool, email sender function via SMTP, and a base exception
class definition. 

\subsubsection{{\publogger} $\ldots$ Logging Module}
This is the framework logger tool, and uses a popular {\python}'s logging module.
{\publogger} is a factory class that can instansiate an individual logger instance
with a specific message format and a choice of stream: either {\stdout}/{\stderr} or
output file stream. 

Each logger instance created by {\publogger} factory has a unique name, and 
can be instantiated by a factory function call:
\begin{lstlisting}
  >>> pubs_logger.get_logger('my_logger')
\end{lstlisting}
for a logger named ``my\_logger''. {\publogger} keeps track of all created loggers 
in its class variable {\ttfamily \_loggers}. When there is a request for a logger
with the same name created in the past, it returns the same instance.
\begin{lstlisting}
  >>> from pub_util import pub_logger
  >>> pub_logger.get_logger('a')
  [ INFO    ] pub_logger (L: 81 ) >> {_add_logger} OPENED LOGGER a
  <logging.Logger object at 0x10b903f90>
  >>> pub_logger.get_logger('a')
  <logging.Logger object at 0x10b903f90>
  >>>
\end{lstlisting}

As you might expect in any similar tool, {\publogger} has several message levels:
debug, info, warning, error, and critical. The default message level is set via
shell environment variable {\ttfamily \$PUB\_LOGGER\_LEVEL}, which is automatically
set in {\ttfamily setup.sh} configuration script. You may change the level if you
wish. You do not have to change the configuration script, but instead just change
the shell environment variable in any way you want (for instance by hand on your
terminal instead of sourcing a script). The set shell environment value is parsed 
in {\ttfamily pub\_util/pub\_env.py} script to an appropriate value. 
Similarly, the stream destination (either {\stdout} or file stream) is
set via shell environment variable {\ttfamily \$PUB\_LOGGER\_DRAIN}, again set
automatically in {\ttfamily setup.sh}. In case the drain is chosen to be a text
file stream, {\ttfamily \$PUB\_LOGGER\_FILE\_LOCATION} environment variable's value
is used as the log file location.

Each message level has a dedicated logger function call to parse an output message
through your logger. Here is an example of formatted output :
\begin{lstlisting}
  >>> a.debug('This is debug')
  [ DEBUG   ] <stdin> (L: 1  ) >> {<module>} This is debug
  >>> a.info('This is info')
  [ INFO    ] <stdin> (L: 1  ) >> {<module>} This is info
  >>> a.warning('This is warning')
  [ WARNING ] <stdin> (L: 1  ) >> {<module>} This is warning
  >>> a.error('This is error')
  [ ERROR   ] <stdin> (L: 1  ) >> {<module>} This is error
  >>> a.critical('This is critical')
  [ CRITICAL] <stdin> (L: 1  ) >> {<module>} This is critical
\end{lstlisting}
The logger specifies the message level, and prints out three more information in
addition to the sent message by the caller. The first `$<$stdin$>$' tells where the
message is sent from. `(L: 1  )' tells us which line in the caller's module code
this function is called from. Then `\{$<$module$>$\}' tells us the name of the
caller's module. In the above example, this is called from the main, and hence
it is not really useful. However, these information help us to track down problems
easily as you can identify where each function call is made. For instance, running
{\ttfamily ds\_daemon.py} to test the installation (see Sec.\ref{prep:pubs:daemon}),
you have probably seen this message:
\begin{lstlisting}
  [ DEBUG   ] ds_daemon (L: 128) >> {load_projects} Updating project dummy_daq ...
\end{lstlisting}
 This menas that a logger function ``debug'' was called by a function 
{\ttfamily load\_projects} and the exact location is in line number 128 of the 
module code {\ttfamily ds\_daemon.py}. 

\subsubsection{{\pubsmtp} $\ldots$ Simple SMTP Protocole}


\subsection{{\pubdbi}}

\subsection{{\dstream}}


\section{{\psql} Database Schema And Functions}

Now that we spent many pages to discuss about the {\pubs} model, let's talk
about something real and practical. This section presents a list of {\pubs} 
functions implemented on the {\psql} server. About a half of them are for
experts' use (in fact mostly for daemon and automated scripts since human
hands are one of last things to be trusted), and the other half is for
project scripts to use. 

If you are a project code developper and do not find a function of your
need, please contact the author and he will be more than happy to assist
how the existing function may solve the problem or implement a brand
new function to make your life easier.

\subsection{Project Information/Status Query}
These are functions that can be used by projects upon execution. That being
said, however, it is {\bf \color{blue} strongly recommended to use {\python} 
API within {\pubs} to execute these functions}. They should not be executed
from {\psql} interpreter or directly executing from an SQL script. If the 
list lacks any function needed for a project execution, please contact the 
author with a request. Functions and corresponding {\python} API will be 
provided.
\begin{itemize}
  \item {\bf DoesTableExist( name TEXT )}
    \begin{itemize}
      \item Checks if a table of the {\it name} exists or not in the database
        by checking the administrative master table. The table name is required
        to be in lowercase (there is no uppercase vs. lowercase in distinction
        among {\psql} server objects).
    \end{itemize}
  \item {\bf DoesProjectExist( name TEXT )}
    \begin{itemize}
      \item Checks if a project with the {\it name} exists or not in the 
        database. In addition to DoesTableExist(), this function checks
        if a specified project exists or not.
    \end{itemize}
  \item {\bf GetRunTimeStamp( Run INT, SubRun INT )}
    \begin{itemize}
      \item A function to retrieve the run start and end time stamp.
    \end{itemize}
  \item {\bf ProjectResource( name TEXT )}
    \begin{itemize}
      \item Returns a project resource (information needed for an execution)
        for a specified project name.
    \end{itemize}
  \item {\bf IncreaseProjSequence( name TEXT, run INT, subrun INT, nseq
    SMALLINT, status SMALLINT) }
    \begin{itemize}
      \item Increase number of sequence count in the specified project table
        for the specified run/sub-run number combination. Input status code 
        is used for all newly created TaskIDs.
    \end{itemize}
  \item {\bf UpdateProjStatus( name TEXT, run INT, subrun INT, seq SMALLINT,
    status SMALLINT, data TEXT)}
    \begin{itemize}
      \item Update the specified project's status for the specified TaskID. 
        At the same time, a TaskID specific data can be also stored although
        that is not necessary (by default the last argument is set to NULL).
    \end{itemize}
  \item {\bf GetProjectData( name TEXT, run INT, subrun INT, seq SMALLINT )}
    \begin{itemize}
      \item Retrieve project data for a specified TaskID. Only accessible to
        The data from the latest version number to avoid a confusion (and
        hence version number cannot be specified).
    \end{itemize}
  \item {\bf GetRuns( name TEXT, status SMALLINT) }
    \begin{itemize}
      \item Returns a table of TaskID (run, sub-run, seq., project-version) 
        for which the specified project carries the specified status code.
    \end{itemize}
  \item {\bf GetRuns( TEXT[]::ARRAY, SMALLINT[]::ARRAY )}
    \begin{itemize}
      \item Similar to GetRuns and it returns a table of run/sub-run number 
      combinations for which all specified projects in the first argument
      carry specified status code in the second argument. This function
      is useful to obtain a list of run/sub-run numbers across multiple
      project tables for specific combination of status code. Because 
      a sequence number is project dependent, it returns run/sub-run for
      which all belonging sequence status uniquely matches with the specified
      status code.
    \end{itemize}

\end{itemize}


\subsection{Functions For Project Management}
These are functions to be used by daemon process to maintaine/running the
projects. In principle these should not be used by a project execution. 
\begin{itemize}
  \item {\bf RemoveProject( name TEXT )} 
    \begin{itemize}
      \item Properly remove a project: drop a project table and remove the
        project information entry from the ProcessTable.
    \end{itemize}
  \item {\bf ListProject()}
    \begin{itemize}
      \item List all projects with the latest version number from ProcessTable.
    \end{itemize}
  \item {\bf ListEnabledProject()}
    \begin{itemize}
      \item List currently enabled project information with the latest version
        number from the ProcessTable.
    \end{itemize}
  \item {\bf DefineProject( name TEXT, command TEXT, frequency INT, email TEXT,
    start\_run INT, start\_subrun INT, resource HSTORE, enabled BOOLEAN )}
    \begin{itemize}
      \item A function to define a new project. It takes in project information
        and registers into the ProcessTable. It also calls {\bf MakeProjTable}
        function to create a project table.
    \end{itemize}
  \item {\bf MakeProjTable( name TEXT )}
    \begin{itemize}
      \item Function dedicated to create a project table. This function is
        to be called by {\bf DefineProject} and not to be called by hand!
    \end{itemize}
  \item {\bf UpdateProjectConfig( name TEXT, command TEXT, frequency INT, email
    TEXT, resource HSTORE, enabled BOOLEAN, version INT)}
    \begin{itemize}
      \item A function to alter and update project configuration. As seen in
        the function arguments, start run/sub-run number cannot be altered by
        design.
    \end{itemize}
  \item {\bf ProjectVersionUpdate( name TEXT, command TEXT, frequency INT,
    email TEXT, run INT, subrun INT, resource HSTORE, enable BOOLEAN)}
    \begin{itemize}
      \item Increment the project version number and store new project
        information. Unlike {\bf UpdateProjectConfig}, this function can
        register any project information as there will be a distinct row
        to be inserted in the ProcessTable.
    \end{itemize}
  \item {\bf GetVersionRunRange( name TEXT )}
    \begin{itemize}
      \item For a specified project name, returns multiple result sets each
        representing a specific run number range with the corresponding
        project version number.
    \end{itemize}
  \item {\bf InsertIntoProjTable( name TEXT, run INT, subrun INT )}
    \begin{itemize}
      \item Insert a new run/sub-run number entry into a project table with
        the default status code of 1. The latest version number for the
        subject run/sub-run is also taken from the ProcessTable.
    \end{itemize}
  \item {\bf OneProjectRunSynch()}
    \begin{itemize}
      \item Make sure one particular project table has run/sub-run 
        numbers that currently appears in the MainRun table and above
        the specified run/subrun numbers in the argument.
    \end{itemize}
  \item {\bf AllProjectRunSynch()}
    \begin{itemize}
      \item Make sure all project table has run/sub-run numbers that 
        currently appears in the MainRun and above the specified start
        run/sub-run numbers in the project information.
    \end{itemize}
  \item {\bf ProjectInfo(name TEXT, ver INT)}
    \begin{itemize}
      \item Returns project information for a specified version number.
        By default the version number does not need to be specified.
        If not given, it is set to the latest version number. This function
        is used to run a project via daemon.
    \end{itemize}
\end{itemize}

\subsection{Admin Functions}
Functions prepared for the top-level administrative purposes. These functions
should be executed by database admins only.
\begin{itemize}
  \item {\bf RemoveProcessDB()}
    \begin{itemize}
      \item ``Properly'' remove {\it everything}. This function drops all
        projects registered in ProcessTable using {\bf RemoveProject} 
        function. Then it drops an empty ProcessTable.
    \end{itemize}
  \item {\bf CreateProcessTable()}
    \begin{itemize}
      \item A simple function to create the ProcessTable.
    \end{itemize}
  \item {\bf CreateTestRunTable()}
    \begin{itemize}
      \item A function to create ``fake'' MainRun table. This is for 
        development work, and not for an official operation. In the official
        production, MainRun table is slave-copied from the configuration
        database automatically.
    \end{itemize}
  \item {\bf InsertIntoTestRunTable( Run INT, SubRun INT, TimeStart TIMESTAMP, TimeEnd TIMESTAMP )}
    \begin{itemize}
      \item A function to insert a new entry into the ``fake'' MainRun table.
        This is not meant to be used for the offial production.
    \end{itemize}
  \item {\bf FillTestRunTable( NRuns INT, NSubRuns INT)}
    \begin{itemize}
      \item A function to fill the ``fake'' MainRun table with multiple entries
        at once. It fills the table with NRuns, each with NSubRuns.
    \end{itemize}
  \item {\bf CheckDBIntegrity()}
    \begin{itemize}
      \item Returns a boolean after checking the process DB integrity. 
        In particular it checks if ProcessTable exists or not, and then
        checks if all projects registered in ProcessTable have own project
        tables.
    \end{itemize}
\end{itemize}





\section{{\php}-based Web Interface}
\input{./src/PUBS/php}




% Bibliography
\bibliographystyle{unsrt}

\bibliography{PUBS}


\end{document}
